% !TEX root = ../Projektdokumentation.tex
\section{Einführungsphase}
\label{sec:Einfuehrungsphase}

\subsection{Anwendungen}
\label{sec:programs_install}

\subsubsection{Oberfläche / \acsu{UI}}
\label{sec:UI}

Die Oberfläche wurde auf dem Cloud-Server der Firma Weiss hochgeladen. Auf diesen Cloud-Server läuft ein Windows-Betriebssystem mit der \acsu{IIS}-Applikation. \\
Die Webanwendungen kann nur aus dem lokalen Netz abgerufen werden, um Unautorisierte Zugriffe zu verhindern. Um Außerhalb der Firma zugreifen zu können, wird ein \acsu{VPN}-Tunnel benötigt. 

\subsubsection{API}
\label{sec:API}

Die \acsu{API} läuft wie die Oberfläche auf dem Cloud-Server der Firma Weiss. Die API ist aus dem öffentlichen Netz erreichbar, da die Daten nur über einen \acsu{SHA}-512 Token freigegeben werden. 

\subsubsection{Verwaltungsdienst}
\label{sec:Service}

Um die Mitarbeiter zu Schulen, wurde auf einem Testgerät der Firma Weiss dieser Verwaltungsdienst installiert. Dieser ist mit einem Testkunde verknüpft.\\
\\
In der Schulung wurden die Dienste auf ausgewählten Kunden-Systeme installiert und ausgeführt.\\
\\
Dadurch das der Dienst sich selbst installieren kann, muss dieser nur ausgeführt werden. Davor muss die Konfiguration angepasst werden: Authentifizierung Token, Datenbank \acsu{ODBC} Daten, Filestore wo die Dienste installiert werden soll. 

\subsection{Schulung}
\label{sec:Schulung}

Die Mitarbeiter wurden in einer 2,5 stündigen Schulungsveranstaltung geschult. Es wurde die Oberfläche präsentiert und erklärt. Danach ist auf die Installation des Verwaltungsdienst eingegangen worden. \\
Die Mitarbeiter mussten selbst auf Kundensysteme den Verwaltungsdienst installieren und konfigurieren. \\
In den letzten 10min ist man auf Fragen und Anregungen eingegangen.
