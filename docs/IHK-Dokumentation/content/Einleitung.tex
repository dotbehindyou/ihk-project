% !TEX root = ../Index.tex
\section{Einleitung}
\label{sec:Einleitung}


\subsection{Projektumfeld} 
\label{sec:Projektumfeld}
Das Projekt wird von Christopher Mogler in der Firma Weiss GmbH Softwarelösungen
entwickelt. Sie wurde 1975 von Rolf Weiss mit Hauptsitz in Schiltach gegründet. Zu den ersten
Anfängen der Firma wurde für mittelständische Unternehmen Auftragsprogrammierungen auf
IBM-Systemen durchgeführt. Im Jahre 1985 wurde die Produktpalette auf \acsu{ERP}-Systeme, für
mittelständische und größere Handels bzw. Industrieunternehmen, erweitert. \\
Die Geschäftsleitung wurde 1998 von Martin Lauble übernommen. Ab dem Jahr 2002 wurde der
Schwerpunkt der Software auf das Produkt PowerWeiss für Windows gelegt, welches als
\acsu{CRM}-System für Handel, Industrie und Versicherungsagenturen genutzt werden kann.

\subsection{Projektziel} 
\label{sec:Projektziel}
Die Firma hat für viele Zusatz Module dazugehörige Dienste. Diese müssen installiert und konfiguriert werden. \\
Der Support muss sich auf die Kunden-Server drauf schalten und danach die Dienste manuell installieren und konfigurieren. Ziel ist es die Installation, Aktualisierung und Deinstallation zu automatisieren.
Deshalb soll ein Dienst programmiert werden, was die anderen Dienste automatisch verwaltet. 
Über eine Oberfläche kann die Firma Weiss, Dienste installieren, konfigurieren, aktualisieren und deinstallieren. Der Dienst auf dem Kunden-Server holt sich die aktuellen Daten über eine \acsu{REST}-\acsu{API} ab. \\
Und gleicht die Dienste mit den Daten der \acsu{API} ab. 
Bei einer Installation sollen die Dateien heruntergeladen werden. Danach werden die Dateien in ein Verzeichnis gespeichert. Die Dienste werden über die Windows-\acsu{API} per \acsu{CS} installiert. Die Dienste werden direkt gestartet.
Bei einer Aktualisierung werden die ausgewählten Dienste gestoppt. Die Dateien werden mit den aktuellen Dateien überschrieben. Danach wird der Dienst wieder gestartet.
Wenn Dienste deinstalliert werden sollen, werden die Dienste gestoppt und über die Windows-\acsu{API} vom Dienst-Verzeichnis entfernt. Die Dateien werden nicht gelöscht, um Fehler vorzubeugen.
\subsection{Projektbegründung} 
\label{sec:Projektbegruendung}
Um Dienste zu installieren muss sich ein Mitarbeiter, der Firma Weiss, auf den Server drauf schalten. 
Die aktuellen Dateien der Dienste kopieren und mit dem Windows bereitgestellten Programm "InstallUtil" installieren. Da "InstallUtil" nur über eine \acsu{CMD} verwendet werden kann, ist es sehr fehleranfällig. \\
Die manuelle Installation ist sehr Zeit aufwendig, deshalb ist eine Automatisierung sehr sinnvoll.
Mit einer Automatisierung werden Dienste per Knopfdruck In-, aktuali- und Deinstalliert oder konfiguriert, ohne das ein Mitarbeiter sich auf einen Server Zugriff verschaffen muss. 

\subsection{Projektschnittstellen} 
\label{sec:Projektschnittstellen}
\begin{itemize}
	\item Mit welchen anderen Systemen interagiert die Anwendung (technische Schnittstellen)?
	\item Wer genehmigt das Projekt \bzw stellt Mittel zur Verfügung? 
	\item Wer sind die Benutzer der Anwendung?
	\item Wem muss das Ergebnis präsentiert werden?
\end{itemize}
Für die Datenspeicherung wird eine SQL-Anywhere-Datenbank verwendet. Der Zugriff auf diese Datenbank, wird über \acsu{ODBC} realisiert. \\
\acsu{UI} und \acsu{REST} werden als Web Applikation bereitgestellt, diese wird über \acsu{ASP} verwaltet. Um \acsu{ASP} zu verwenden wird ein \acsu{IIS}-Dienst von Windows benötigt. \\
Um Windows-Dienste verwalten zu können, wird auf die Windows-\acsu{API} zugegriffen, über die AdvAPI32 Bibliothek. \\
\acsu{UI} verwendet \acsu{React} als Framework und \acsu{Node} zur verwaltung von \acsu{React}. 
\\
Das Projekt wurde durch Herrn Richter genehmigt, der Abteilungsleiter der Entwicklung – bei der Firma Weiss. Die Endnutzer sind die Mitarbeiter der Firma Weiss, die geschult werden durch den Autor und eine Anwenderdokumentation erhalten. \\

\subsection{Projektabgrenzung[TODO]} 
\label{sec:Projektabgrenzung}
\begin{itemize}
	\item Was ist explizit nicht Teil des Projekts (\insb bei Teilprojekten)?
\end{itemize}
